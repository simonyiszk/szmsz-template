%%%%%%%%%%%%%%%%%%%%%%%%%%%%%%%%%%%%%%%%%%%%%%%%%%%%%%%%%%%%%%%%%%
%                           Document style
%%%%%%%%%%%%%%%%%%%%%%%%%%%%%%%%%%%%%%%%%%%%%%%%%%%%%%%%%%%%%%%%%%
\documentclass[12pt]{article}

% Importing report style and the first pages
%%%%%%%%%%%%%%%%%%%%%%%%%%%%%%%%%%%%%%%%%%%%%%%%%%%%%%%%%%%%%%%%%%
%                           Document style
%%%%%%%%%%%%%%%%%%%%%%%%%%%%%%%%%%%%%%%%%%%%%%%%%%%%%%%%%%%%%%%%%%
\documentclass[12pt]{article}

% Importing report style and the first pages
%%%%%%%%%%%%%%%%%%%%%%%%%%%%%%%%%%%%%%%%%%%%%%%%%%%%%%%%%%%%%%%%%%
%                           Document style
%%%%%%%%%%%%%%%%%%%%%%%%%%%%%%%%%%%%%%%%%%%%%%%%%%%%%%%%%%%%%%%%%%
\documentclass[12pt]{article}

% Importing report style and the first pages
%%%%%%%%%%%%%%%%%%%%%%%%%%%%%%%%%%%%%%%%%%%%%%%%%%%%%%%%%%%%%%%%%%
%                           Document style
%%%%%%%%%%%%%%%%%%%%%%%%%%%%%%%%%%%%%%%%%%%%%%%%%%%%%%%%%%%%%%%%%%
\documentclass[12pt]{article}

% Importing report style and the first pages
\input{./kir-dev.sty}

% sections without numbers, but still in table of contents
%\setcounter{secnumdepth}{0} % remove "%" to include

%=================================================================
%                           Start Document
%=================================================================
\begin{document}

% adding the first two pages
\FirstPage
\pagebreak

A Budapesti Műszaki és Gazdaságtudományi Egyetem Villamosmérnöki és Informatikai Kar Schönherzes 
Villamosmérnökök és Informatikusok Egyesületének (továbbiakban: Egyesület) a Körök alapítására 
vonatkozó szabályozása alapján a Kollégiumi Információs Rendszer Fejlesztők és Üzemeltetők 
Szervezeti és Működési Szabályzatát (továbbiakban: SZMSZ) az alábbiakban állapítja meg:

\section{A Kör alapadatai}

\begin{enumerate}
  \item A Kör neve: Kollégiumi Információs Rendszer Fejlesztők és Üzemeltetők
  \item A Kör Reszortja: Simonyi Károly Szakkollégium
  \item A Kör hivatalos elérhetősége: kir-dev@sch.bme.hu
  \item A Kör alapításának éve: 2001
\end{enumerate}

% Section border ==========================================================

\section{A Kör alapvető tevékenysége}

A Kör célja a kar és a Schönherz Kollégium számára hasznos web- és mobilalkalmazások fejlesztése, 
valamint közös informatikai érdeklődéssel rendelkező egyetemi hallgatókat egybegyűjteni.

% Section border ==========================================================

\section{A Kör felépítése}

\begin{enumerate}
  \item A Kör tagságára jelentkezhet minden rendes és külső Egyesületi tag, aki elfogadja és magára nézve kötelezőnek ismeri el a Kör SZMSZ-ét, a Reszort SZMSZ-ét, valamint egyetért a Kör céljaival és feladataival.
  
  \item A Kör tagja az lehet, aki teljesíti a Kör felvételi kritériumait.

  \item A Kör felvételi kritériumai:
    \begin{enumerate}
    \item A Kör Felvételi Bizottsága által meghatározott aktuális követelmények.
    \end{enumerate}

  \item A Kör posztjai

\begin{enumerate} % posztok
  \item Körvezető
    \begin{enumerate}
    \item Feladatairól a 4. § rendelkezik.
    \end{enumerate}

  \item Gazdasági felelős
    \begin{enumerate}
        \item Feladatairól a 4. § rendelkezik.
    \end{enumerate}

  \item Körtag
    \begin{enumerate}
        \item Feladata a Körön belüli munkavégzés és a Körhöz méltó viselkedés.
    \end{enumerate}

  \item Öregtag
    \begin{enumerate}
        \item Feladattal nem rendelkezik, tanácsadó szerepet tölt be.
    \end{enumerate}

  \item Újonc
    \begin{enumerate}
        \item Feladata a Kör megismerése és projekthez csatlakozás.
    \end{enumerate}

  \item PéK admin
    \begin{enumerate}
        \item Feladata a SVIE adminisztrációs rendszerének (PéK) adminisztratív ügyeinek intézése.
    \end{enumerate}

  \item Oktatási felelős
    \begin{enumerate}
        \item Feladata a Kör tanfolyamainak, workshopjainak megszervezése és a Körben folytatott oktatási tevékenységek felügyelete.
    \end{enumerate}

  \item Szerver admin
    \begin{enumerate}
        \item Feladata a szerverek karbantartása és üzemeltetése.
    \end{enumerate}

  \item PR felelős
    \begin{enumerate}
        \item Feladata a Kör külső megjelenéseinek szervezése.
    \end{enumerate}

  \item Mentor
    \begin{enumerate}
      \item Feladata az újoncok munkájának segítése, beilleszkedés megkönnyítése.
    \end{enumerate}

\end{enumerate} % end of posztok

% +-+-+-+-+-+ posztok vége, de még felépítés

\item Poszt betöltésének kritériumai:

  \begin{enumerate}
    \item A Körvezető, illetve Gazdasági felelős választásáról a 4-6. § rendelkezik.
  \end{enumerate}

\item A Körtagság megszűnhet

  \begin{enumerate}
    \item kilépéssel,
    \item egyesületi tagság megszűnésével,
    \item ha a Felügyelő Bizottság, az Elnökség, a Reszortvezetők Tanácsa vagy a Kör a 6. § által meghatározott módon úgy dönt.
  \end{enumerate}

\end{enumerate} % felépítés vége

% Section border ==========================================================

\section{A Kör vezetője}

\begin{enumerate}
  \item A Kör vezetőjét évente a Kör egy évre választja a 6. § által meghatározott módon.
  \item A Körvezető felelős a Kör eszközeinek állapotmegóvásáért, valamint a Kör leltárának helyességéért.
  \item A Körvezető felelős a Kör által szervezett események idején a Körtagok tevékenységéért.
  \item A Körvezető kötelessége a Körtagokat tájékoztatni az Egyesületi döntésekről és változásokról.
  \item A Körvezető kötelessége vezetőváltás esetén az új Körvezetőt megismertetni kötelezettségeivel.
  \item A Körvezető kötelessége kinevezni a Kör mentorait.
\end{enumerate}

% Section border ==========================================================

\section{A Kör gazdálkodása}

\begin{enumerate}
  \item A Kör gazdasági felelősét a Körvezető nevezi ki.
  \item A Kör gazdasági felelősének feladatai:
  \begin{enumerate}
    \item a Kör éves gazdasági pályázatának megtervezése és elkészítése a Kör érdekeinek megfelelően,
    \item a Kör évközi költéseinek felügyelete,
    \item a Kör éves gazdasági beszámolóinak megírása.
  \end{enumerate}
\end{enumerate}

% Section border ==========================================================

\section{Döntéshozatal a Körben}

\begin{enumerate}

  \item A Kör döntése érvényes, amennyiben legalább a Körtagok fele részt vett annak meghozásában.
  \item A Kör döntéseit körgyűlés alapján egyszerű többséggel hozza:
  \begin{enumerate}
    \item Küldöttek állítása,
    \item Körtagok kizárása,
    \item Gazdasági pályázat elfogadása,
    \item Felvételi Bizottság tagjainak kiválasztása,
    \item Tanfolyamfelelős elfogadása,
    \item PéK adminok elfogadása,
    \item Szerver adminok elfogadása,
    \item PR felelős elfogadása,
    \item Körtag posztjainak megvonása, a Körvezetői poszt kivételével.
  \end{enumerate}
  \item A Körtagok legalább kétharmados többsége szükséges:
  \begin{enumerate}
    \item Körvezető választásához,
    \item SZMSZ módosításához,
    \item Kör megszüntetéséhez.
  \end{enumerate}
  \item A kör felvételijével kapcsolatos döntéseket a kör azon tagjai hozzák, akik tagjai a Felvételi Bizottságnak. A felvételi döntéseket a Felvételi Bizottság a gyűlésein tett szavazatok alapján egyszerű többséggel hozza:
  \begin{enumerate}
    \item Az adott Újonc felvételének elfogadása, a Körbe történő felvétele.
  \end{enumerate}
  
\end{enumerate}

% Section border ==========================================================

\section{Záró rendelkezések}

\begin{enumerate}
\item Jelen szabályzat által nem érintett kérdésekben a Reszort SZMSZ-e, az Egyesület Alapszabálya, valamint más vonatkozó jogszabályok és rendeletek az irányadók.
\item Jelen szabályzatot a Reszortvezetők Tanácsa fogadja el, a Kör vezetőjének javaslatára.
\item A szabályzat a Reszortvezetők Tanácsának megszavazása által lép hatályba. Ezzel egyidőben hatályát veszíti a Kör korábban elfogadott Szabályzata.
\item Jelen Szervezeti és Működési Szabályzatot a Kör tagjai akaratukkal megegyezőként fogadják el.
\end{enumerate}

%=================================================================
%                           End Document
%=================================================================
\end{document}


% sections without numbers, but still in table of contents
%\setcounter{secnumdepth}{0} % remove "%" to include

%=================================================================
%                           Start Document
%=================================================================
\begin{document}

% adding the first two pages
\FirstPage
\pagebreak

A Budapesti Műszaki és Gazdaságtudományi Egyetem Villamosmérnöki és Informatikai Kar Schönherzes 
Villamosmérnökök és Informatikusok Egyesületének (továbbiakban: Egyesület) a Körök alapítására 
vonatkozó szabályozása alapján a Kollégiumi Információs Rendszer Fejlesztők és Üzemeltetők 
Szervezeti és Működési Szabályzatát (továbbiakban: SZMSZ) az alábbiakban állapítja meg:

\section{A Kör alapadatai}

\begin{enumerate}
  \item A Kör neve: Kollégiumi Információs Rendszer Fejlesztők és Üzemeltetők
  \item A Kör Reszortja: Simonyi Károly Szakkollégium
  \item A Kör hivatalos elérhetősége: kir-dev@sch.bme.hu
  \item A Kör alapításának éve: 2001
\end{enumerate}

% Section border ==========================================================

\section{A Kör alapvető tevékenysége}

A Kör célja a kar és a Schönherz Kollégium számára hasznos web- és mobilalkalmazások fejlesztése, 
valamint közös informatikai érdeklődéssel rendelkező egyetemi hallgatókat egybegyűjteni.

% Section border ==========================================================

\section{A Kör felépítése}

\begin{enumerate}
  \item A Kör tagságára jelentkezhet minden rendes és külső Egyesületi tag, aki elfogadja és magára nézve kötelezőnek ismeri el a Kör SZMSZ-ét, a Reszort SZMSZ-ét, valamint egyetért a Kör céljaival és feladataival.
  
  \item A Kör tagja az lehet, aki teljesíti a Kör felvételi kritériumait.

  \item A Kör felvételi kritériumai:
    \begin{enumerate}
    \item A Kör Felvételi Bizottsága által meghatározott aktuális követelmények.
    \end{enumerate}

  \item A Kör posztjai

\begin{enumerate} % posztok
  \item Körvezető
    \begin{enumerate}
    \item Feladatairól a 4. § rendelkezik.
    \end{enumerate}

  \item Gazdasági felelős
    \begin{enumerate}
        \item Feladatairól a 4. § rendelkezik.
    \end{enumerate}

  \item Körtag
    \begin{enumerate}
        \item Feladata a Körön belüli munkavégzés és a Körhöz méltó viselkedés.
    \end{enumerate}

  \item Öregtag
    \begin{enumerate}
        \item Feladattal nem rendelkezik, tanácsadó szerepet tölt be.
    \end{enumerate}

  \item Újonc
    \begin{enumerate}
        \item Feladata a Kör megismerése és projekthez csatlakozás.
    \end{enumerate}

  \item PéK admin
    \begin{enumerate}
        \item Feladata a SVIE adminisztrációs rendszerének (PéK) adminisztratív ügyeinek intézése.
    \end{enumerate}

  \item Oktatási felelős
    \begin{enumerate}
        \item Feladata a Kör tanfolyamainak, workshopjainak megszervezése és a Körben folytatott oktatási tevékenységek felügyelete.
    \end{enumerate}

  \item Szerver admin
    \begin{enumerate}
        \item Feladata a szerverek karbantartása és üzemeltetése.
    \end{enumerate}

  \item PR felelős
    \begin{enumerate}
        \item Feladata a Kör külső megjelenéseinek szervezése.
    \end{enumerate}

  \item Mentor
    \begin{enumerate}
      \item Feladata az újoncok munkájának segítése, beilleszkedés megkönnyítése.
    \end{enumerate}

\end{enumerate} % end of posztok

% +-+-+-+-+-+ posztok vége, de még felépítés

\item Poszt betöltésének kritériumai:

  \begin{enumerate}
    \item A Körvezető, illetve Gazdasági felelős választásáról a 4-6. § rendelkezik.
  \end{enumerate}

\item A Körtagság megszűnhet

  \begin{enumerate}
    \item kilépéssel,
    \item egyesületi tagság megszűnésével,
    \item ha a Felügyelő Bizottság, az Elnökség, a Reszortvezetők Tanácsa vagy a Kör a 6. § által meghatározott módon úgy dönt.
  \end{enumerate}

\end{enumerate} % felépítés vége

% Section border ==========================================================

\section{A Kör vezetője}

\begin{enumerate}
  \item A Kör vezetőjét évente a Kör egy évre választja a 6. § által meghatározott módon.
  \item A Körvezető felelős a Kör eszközeinek állapotmegóvásáért, valamint a Kör leltárának helyességéért.
  \item A Körvezető felelős a Kör által szervezett események idején a Körtagok tevékenységéért.
  \item A Körvezető kötelessége a Körtagokat tájékoztatni az Egyesületi döntésekről és változásokról.
  \item A Körvezető kötelessége vezetőváltás esetén az új Körvezetőt megismertetni kötelezettségeivel.
  \item A Körvezető kötelessége kinevezni a Kör mentorait.
\end{enumerate}

% Section border ==========================================================

\section{A Kör gazdálkodása}

\begin{enumerate}
  \item A Kör gazdasági felelősét a Körvezető nevezi ki.
  \item A Kör gazdasági felelősének feladatai:
  \begin{enumerate}
    \item a Kör éves gazdasági pályázatának megtervezése és elkészítése a Kör érdekeinek megfelelően,
    \item a Kör évközi költéseinek felügyelete,
    \item a Kör éves gazdasági beszámolóinak megírása.
  \end{enumerate}
\end{enumerate}

% Section border ==========================================================

\section{Döntéshozatal a Körben}

\begin{enumerate}

  \item A Kör döntése érvényes, amennyiben legalább a Körtagok fele részt vett annak meghozásában.
  \item A Kör döntéseit körgyűlés alapján egyszerű többséggel hozza:
  \begin{enumerate}
    \item Küldöttek állítása,
    \item Körtagok kizárása,
    \item Gazdasági pályázat elfogadása,
    \item Felvételi Bizottság tagjainak kiválasztása,
    \item Tanfolyamfelelős elfogadása,
    \item PéK adminok elfogadása,
    \item Szerver adminok elfogadása,
    \item PR felelős elfogadása,
    \item Körtag posztjainak megvonása, a Körvezetői poszt kivételével.
  \end{enumerate}
  \item A Körtagok legalább kétharmados többsége szükséges:
  \begin{enumerate}
    \item Körvezető választásához,
    \item SZMSZ módosításához,
    \item Kör megszüntetéséhez.
  \end{enumerate}
  \item A kör felvételijével kapcsolatos döntéseket a kör azon tagjai hozzák, akik tagjai a Felvételi Bizottságnak. A felvételi döntéseket a Felvételi Bizottság a gyűlésein tett szavazatok alapján egyszerű többséggel hozza:
  \begin{enumerate}
    \item Az adott Újonc felvételének elfogadása, a Körbe történő felvétele.
  \end{enumerate}
  
\end{enumerate}

% Section border ==========================================================

\section{Záró rendelkezések}

\begin{enumerate}
\item Jelen szabályzat által nem érintett kérdésekben a Reszort SZMSZ-e, az Egyesület Alapszabálya, valamint más vonatkozó jogszabályok és rendeletek az irányadók.
\item Jelen szabályzatot a Reszortvezetők Tanácsa fogadja el, a Kör vezetőjének javaslatára.
\item A szabályzat a Reszortvezetők Tanácsának megszavazása által lép hatályba. Ezzel egyidőben hatályát veszíti a Kör korábban elfogadott Szabályzata.
\item Jelen Szervezeti és Működési Szabályzatot a Kör tagjai akaratukkal megegyezőként fogadják el.
\end{enumerate}

%=================================================================
%                           End Document
%=================================================================
\end{document}


% sections without numbers, but still in table of contents
%\setcounter{secnumdepth}{0} % remove "%" to include

%=================================================================
%                           Start Document
%=================================================================
\begin{document}

% adding the first two pages
\FirstPage
\pagebreak

A Budapesti Műszaki és Gazdaságtudományi Egyetem Villamosmérnöki és Informatikai Kar Schönherzes 
Villamosmérnökök és Informatikusok Egyesületének (továbbiakban: Egyesület) a Körök alapítására 
vonatkozó szabályozása alapján a Kollégiumi Információs Rendszer Fejlesztők és Üzemeltetők 
Szervezeti és Működési Szabályzatát (továbbiakban: SZMSZ) az alábbiakban állapítja meg:

\section{A Kör alapadatai}

\begin{enumerate}
  \item A Kör neve: Kollégiumi Információs Rendszer Fejlesztők és Üzemeltetők
  \item A Kör Reszortja: Simonyi Károly Szakkollégium
  \item A Kör hivatalos elérhetősége: kir-dev@sch.bme.hu
  \item A Kör alapításának éve: 2001
\end{enumerate}

% Section border ==========================================================

\section{A Kör alapvető tevékenysége}

A Kör célja a kar és a Schönherz Kollégium számára hasznos web- és mobilalkalmazások fejlesztése, 
valamint közös informatikai érdeklődéssel rendelkező egyetemi hallgatókat egybegyűjteni.

% Section border ==========================================================

\section{A Kör felépítése}

\begin{enumerate}
  \item A Kör tagságára jelentkezhet minden rendes és külső Egyesületi tag, aki elfogadja és magára nézve kötelezőnek ismeri el a Kör SZMSZ-ét, a Reszort SZMSZ-ét, valamint egyetért a Kör céljaival és feladataival.
  
  \item A Kör tagja az lehet, aki teljesíti a Kör felvételi kritériumait.

  \item A Kör felvételi kritériumai:
    \begin{enumerate}
    \item A Kör Felvételi Bizottsága által meghatározott aktuális követelmények.
    \end{enumerate}

  \item A Kör posztjai

\begin{enumerate} % posztok
  \item Körvezető
    \begin{enumerate}
    \item Feladatairól a 4. § rendelkezik.
    \end{enumerate}

  \item Gazdasági felelős
    \begin{enumerate}
        \item Feladatairól a 4. § rendelkezik.
    \end{enumerate}

  \item Körtag
    \begin{enumerate}
        \item Feladata a Körön belüli munkavégzés és a Körhöz méltó viselkedés.
    \end{enumerate}

  \item Öregtag
    \begin{enumerate}
        \item Feladattal nem rendelkezik, tanácsadó szerepet tölt be.
    \end{enumerate}

  \item Újonc
    \begin{enumerate}
        \item Feladata a Kör megismerése és projekthez csatlakozás.
    \end{enumerate}

  \item PéK admin
    \begin{enumerate}
        \item Feladata a SVIE adminisztrációs rendszerének (PéK) adminisztratív ügyeinek intézése.
    \end{enumerate}

  \item Oktatási felelős
    \begin{enumerate}
        \item Feladata a Kör tanfolyamainak, workshopjainak megszervezése és a Körben folytatott oktatási tevékenységek felügyelete.
    \end{enumerate}

  \item Szerver admin
    \begin{enumerate}
        \item Feladata a szerverek karbantartása és üzemeltetése.
    \end{enumerate}

  \item PR felelős
    \begin{enumerate}
        \item Feladata a Kör külső megjelenéseinek szervezése.
    \end{enumerate}

  \item Mentor
    \begin{enumerate}
      \item Feladata az újoncok munkájának segítése, beilleszkedés megkönnyítése.
    \end{enumerate}

\end{enumerate} % end of posztok

% +-+-+-+-+-+ posztok vége, de még felépítés

\item Poszt betöltésének kritériumai:

  \begin{enumerate}
    \item A Körvezető, illetve Gazdasági felelős választásáról a 4-6. § rendelkezik.
  \end{enumerate}

\item A Körtagság megszűnhet

  \begin{enumerate}
    \item kilépéssel,
    \item egyesületi tagság megszűnésével,
    \item ha a Felügyelő Bizottság, az Elnökség, a Reszortvezetők Tanácsa vagy a Kör a 6. § által meghatározott módon úgy dönt.
  \end{enumerate}

\end{enumerate} % felépítés vége

% Section border ==========================================================

\section{A Kör vezetője}

\begin{enumerate}
  \item A Kör vezetőjét évente a Kör egy évre választja a 6. § által meghatározott módon.
  \item A Körvezető felelős a Kör eszközeinek állapotmegóvásáért, valamint a Kör leltárának helyességéért.
  \item A Körvezető felelős a Kör által szervezett események idején a Körtagok tevékenységéért.
  \item A Körvezető kötelessége a Körtagokat tájékoztatni az Egyesületi döntésekről és változásokról.
  \item A Körvezető kötelessége vezetőváltás esetén az új Körvezetőt megismertetni kötelezettségeivel.
  \item A Körvezető kötelessége kinevezni a Kör mentorait.
\end{enumerate}

% Section border ==========================================================

\section{A Kör gazdálkodása}

\begin{enumerate}
  \item A Kör gazdasági felelősét a Körvezető nevezi ki.
  \item A Kör gazdasági felelősének feladatai:
  \begin{enumerate}
    \item a Kör éves gazdasági pályázatának megtervezése és elkészítése a Kör érdekeinek megfelelően,
    \item a Kör évközi költéseinek felügyelete,
    \item a Kör éves gazdasági beszámolóinak megírása.
  \end{enumerate}
\end{enumerate}

% Section border ==========================================================

\section{Döntéshozatal a Körben}

\begin{enumerate}

  \item A Kör döntése érvényes, amennyiben legalább a Körtagok fele részt vett annak meghozásában.
  \item A Kör döntéseit körgyűlés alapján egyszerű többséggel hozza:
  \begin{enumerate}
    \item Küldöttek állítása,
    \item Körtagok kizárása,
    \item Gazdasági pályázat elfogadása,
    \item Felvételi Bizottság tagjainak kiválasztása,
    \item Tanfolyamfelelős elfogadása,
    \item PéK adminok elfogadása,
    \item Szerver adminok elfogadása,
    \item PR felelős elfogadása,
    \item Körtag posztjainak megvonása, a Körvezetői poszt kivételével.
  \end{enumerate}
  \item A Körtagok legalább kétharmados többsége szükséges:
  \begin{enumerate}
    \item Körvezető választásához,
    \item SZMSZ módosításához,
    \item Kör megszüntetéséhez.
  \end{enumerate}
  \item A kör felvételijével kapcsolatos döntéseket a kör azon tagjai hozzák, akik tagjai a Felvételi Bizottságnak. A felvételi döntéseket a Felvételi Bizottság a gyűlésein tett szavazatok alapján egyszerű többséggel hozza:
  \begin{enumerate}
    \item Az adott Újonc felvételének elfogadása, a Körbe történő felvétele.
  \end{enumerate}
  
\end{enumerate}

% Section border ==========================================================

\section{Záró rendelkezések}

\begin{enumerate}
\item Jelen szabályzat által nem érintett kérdésekben a Reszort SZMSZ-e, az Egyesület Alapszabálya, valamint más vonatkozó jogszabályok és rendeletek az irányadók.
\item Jelen szabályzatot a Reszortvezetők Tanácsa fogadja el, a Kör vezetőjének javaslatára.
\item A szabályzat a Reszortvezetők Tanácsának megszavazása által lép hatályba. Ezzel egyidőben hatályát veszíti a Kör korábban elfogadott Szabályzata.
\item Jelen Szervezeti és Működési Szabályzatot a Kör tagjai akaratukkal megegyezőként fogadják el.
\end{enumerate}

%=================================================================
%                           End Document
%=================================================================
\end{document}


% sections without numbers, but still in table of contents
%\setcounter{secnumdepth}{0} % remove "%" to include

%=================================================================
%                           Start Document
%=================================================================
\begin{document}

% adding the first two pages
\FirstPage
\pagebreak

A Budapesti Műszaki és Gazdaságtudományi Egyetem Villamosmérnöki és Informatikai Kar Schönherzes 
Villamosmérnökök és Informatikusok Egyesületének (továbbiakban: Egyesület) a Körök alapítására 
vonatkozó szabályozása alapján a Kollégiumi Információs Rendszer Fejlesztők és Üzemeltetők 
Szervezeti és Működési Szabályzatát (továbbiakban: SZMSZ) az alábbiakban állapítja meg:

\section{A Kör alapadatai}

\begin{enumerate}
  \item A Kör neve: Kollégiumi Információs Rendszer Fejlesztők és Üzemeltetők
  \item A Kör Reszortja: Simonyi Károly Szakkollégium
  \item A Kör hivatalos elérhetősége: kir-dev@sch.bme.hu
  \item A Kör alapításának éve: 2001
\end{enumerate}

% Section border ==========================================================

\section{A Kör alapvető tevékenysége}

A Kör célja a kar és a Schönherz Kollégium számára hasznos web- és mobilalkalmazások fejlesztése, 
valamint közös informatikai érdeklődéssel rendelkező egyetemi hallgatókat egybegyűjteni.

% Section border ==========================================================

\section{A Kör felépítése}

\begin{enumerate}
  \item A Kör tagságára jelentkezhet minden rendes és külső Egyesületi tag, aki elfogadja és magára nézve kötelezőnek ismeri el a Kör SZMSZ-ét, a Reszort SZMSZ-ét, valamint egyetért a Kör céljaival és feladataival.
  
  \item A Kör tagja az lehet, aki teljesíti a Kör felvételi kritériumait.

  \item A Kör felvételi kritériumai:
    \begin{enumerate}
    \item A Kör Felvételi Bizottsága által meghatározott aktuális követelmények.
    \end{enumerate}

  \item A Kör posztjai

\begin{enumerate} % posztok
  \item Körvezető
    \begin{enumerate}
    \item Feladatairól a 4. § rendelkezik.
    \end{enumerate}

  \item Gazdasági felelős
    \begin{enumerate}
        \item Feladatairól a 4. § rendelkezik.
    \end{enumerate}

  \item Körtag
    \begin{enumerate}
        \item Feladata a Körön belüli munkavégzés és a Körhöz méltó viselkedés.
    \end{enumerate}

  \item Öregtag
    \begin{enumerate}
        \item Feladattal nem rendelkezik, tanácsadó szerepet tölt be.
    \end{enumerate}

  \item Újonc
    \begin{enumerate}
        \item Feladata a Kör megismerése és projekthez csatlakozás.
    \end{enumerate}

  \item PéK admin
    \begin{enumerate}
        \item Feladata a SVIE adminisztrációs rendszerének (PéK) adminisztratív ügyeinek intézése.
    \end{enumerate}

  \item Oktatási felelős
    \begin{enumerate}
        \item Feladata a Kör tanfolyamainak, workshopjainak megszervezése és a Körben folytatott oktatási tevékenységek felügyelete.
    \end{enumerate}

  \item Szerver admin
    \begin{enumerate}
        \item Feladata a szerverek karbantartása és üzemeltetése.
    \end{enumerate}

  \item PR felelős
    \begin{enumerate}
        \item Feladata a Kör külső megjelenéseinek szervezése.
    \end{enumerate}

  \item Mentor
    \begin{enumerate}
      \item Feladata az újoncok munkájának segítése, beilleszkedés megkönnyítése.
    \end{enumerate}

\end{enumerate} % end of posztok

% +-+-+-+-+-+ posztok vége, de még felépítés

\item Poszt betöltésének kritériumai:

  \begin{enumerate}
    \item A Körvezető, illetve Gazdasági felelős választásáról a 4-6. § rendelkezik.
  \end{enumerate}

\item A Körtagság megszűnhet

  \begin{enumerate}
    \item kilépéssel,
    \item egyesületi tagság megszűnésével,
    \item ha a Felügyelő Bizottság, az Elnökség, a Reszortvezetők Tanácsa vagy a Kör a 6. § által meghatározott módon úgy dönt.
  \end{enumerate}

\end{enumerate} % felépítés vége

% Section border ==========================================================

\section{A Kör vezetője}

\begin{enumerate}
  \item A Kör vezetőjét évente a Kör egy évre választja a 6. § által meghatározott módon.
  \item A Körvezető felelős a Kör eszközeinek állapotmegóvásáért, valamint a Kör leltárának helyességéért.
  \item A Körvezető felelős a Kör által szervezett események idején a Körtagok tevékenységéért.
  \item A Körvezető kötelessége a Körtagokat tájékoztatni az Egyesületi döntésekről és változásokról.
  \item A Körvezető kötelessége vezetőváltás esetén az új Körvezetőt megismertetni kötelezettségeivel.
  \item A Körvezető kötelessége kinevezni a Kör mentorait.
\end{enumerate}

% Section border ==========================================================

\section{A Kör gazdálkodása}

\begin{enumerate}
  \item A Kör gazdasági felelősét a Körvezető nevezi ki.
  \item A Kör gazdasági felelősének feladatai:
  \begin{enumerate}
    \item a Kör éves gazdasági pályázatának megtervezése és elkészítése a Kör érdekeinek megfelelően,
    \item a Kör évközi költéseinek felügyelete,
    \item a Kör éves gazdasági beszámolóinak megírása.
  \end{enumerate}
\end{enumerate}

% Section border ==========================================================

\section{Döntéshozatal a Körben}

\begin{enumerate}

  \item A Kör döntése érvényes, amennyiben legalább a Körtagok fele részt vett annak meghozásában.
  \item A Kör döntéseit körgyűlés alapján egyszerű többséggel hozza:
  \begin{enumerate}
    \item Küldöttek állítása,
    \item Körtagok kizárása,
    \item Gazdasági pályázat elfogadása,
    \item Felvételi Bizottság tagjainak kiválasztása,
    \item Tanfolyamfelelős elfogadása,
    \item PéK adminok elfogadása,
    \item Szerver adminok elfogadása,
    \item PR felelős elfogadása,
    \item Körtag posztjainak megvonása, a Körvezetői poszt kivételével.
  \end{enumerate}
  \item A Körtagok legalább kétharmados többsége szükséges:
  \begin{enumerate}
    \item Körvezető választásához,
    \item SZMSZ módosításához,
    \item Kör megszüntetéséhez.
  \end{enumerate}
  \item A kör felvételijével kapcsolatos döntéseket a kör azon tagjai hozzák, akik tagjai a Felvételi Bizottságnak. A felvételi döntéseket a Felvételi Bizottság a gyűlésein tett szavazatok alapján egyszerű többséggel hozza:
  \begin{enumerate}
    \item Az adott Újonc felvételének elfogadása, a Körbe történő felvétele.
  \end{enumerate}
  
\end{enumerate}

% Section border ==========================================================

\section{Záró rendelkezések}

\begin{enumerate}
\item Jelen szabályzat által nem érintett kérdésekben a Reszort SZMSZ-e, az Egyesület Alapszabálya, valamint más vonatkozó jogszabályok és rendeletek az irányadók.
\item Jelen szabályzatot a Reszortvezetők Tanácsa fogadja el, a Kör vezetőjének javaslatára.
\item A szabályzat a Reszortvezetők Tanácsának megszavazása által lép hatályba. Ezzel egyidőben hatályát veszíti a Kör korábban elfogadott Szabályzata.
\item Jelen Szervezeti és Működési Szabályzatot a Kör tagjai akaratukkal megegyezőként fogadják el.
\end{enumerate}

%=================================================================
%                           End Document
%=================================================================
\end{document}
